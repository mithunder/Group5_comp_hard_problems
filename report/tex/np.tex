
\section{b) NP}

\textit{Show that LazyTSP is in NP.}

To show that LazyTSP is in NP, it must be shown that
there exists a polynomial-time algorithm A that takes
as input a problem instance of LazyTSP and a random string
and determines whether the random string is a solution for
LazyTSP. Basically, whether or not a suggested solution
can be checked in polynomial time.

\paragraph{Algorithm}

Below, the algorithm A will be described:

\begin{enumerate}
\item	Calculate the number \(v = (n - p - m - 1)*(log_2(n)+2\) and check the bitstring.
		If it is below v bits, skip it.
\item	Split the first v bits into (n - p - m - 1) blocks of equal size, and interpret each block
		as a number, specifically a node index. Check that no number occurs twice.
\item	Check that every number is in the range \(m+2..n\).
\item	Interpret the (n - p - m -1) as \(i_2..i_{n-k-m}\), and calculate the left-hand side from \(\star\).
		If the value is below or equal to B, return YES. Else, return NO.
\end{enumerate}

\paragraph{Running time}

In this paragraph, the running time will be shown to be polynomial.

Step 1 takes linear time in n to check the number of bits.
Step 2 takes linear time in n to split each block of bits and to
convert each block. Checking that no number occurs twice takes square time in n.
Step 3 takes linear time.
Step 4 takes constant time.

Since the number of steps are constant, and each step takes polynomial time in n,
the algorithm is polynomial.

\paragraph{Probability of YES and NO}

In this paragraph, the probability of YES if the problem instance has a solution
will be shown to be non-zero, and the probability of NO if the problem instance
does not have a solution will be shown to be 1.

YES: It is assumed that there exists some solution. Consider some soluion.
This solution can be described as a bitstring encoding the solution in the form
of \(i_2..i_{n-k-m}\), in which each index is a block of size \(log_2(n)+2\).
When run through the algorithm, this solution will pass step 1, 2 and 3.
When it comes to step 4, the value will be calculated, and per definition,
since it is a solution, it will fullfil the constraint given in \(\star\),
after which YES will be returned. Since the input string randomly just
might happen to be of the solutions form, the probability for YES is positive,
and thus non-zero.

NO: Assume that there does not exist any solution. If the bitstring does not
pass step 1, 2 and 3, NO will be returned. If the bitstring does pass those
steps, it is a valid guess. Now, since there does not exist any solution,
it will not fulfill the constraint \(\star\), and step 4 will calculate a value
which is larger than B, and then return NO. Thus, if there is no solution
to the problem instance, NO will always be returned.

