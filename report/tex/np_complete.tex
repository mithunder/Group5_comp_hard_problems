
\section{c) NP-complete}

\textit{Show that LazyTSP is NP-complete. What can you say about the complexity
of the problem if you know that \(k = n - a\) or \(m = n - a\) for some a \(\in N\), which
does not depend on n?}

\subsection{Proving NP-completeness}

In b) it was shown that LazyTSP was in NP.
To show that LazyTSP is NP-complete, it only needs to be shown that
it is at least as hard as another NP-complete problem. TSP seems to be a suitable
candidate for this. If a problem instance of TSP can be converted to
a problem instance of LazyTSP in polynomial time, and the YES-NO-property
does not change, then it has been shown that LazyTSP is at least as hard
as TSP, and is thus NP-complete.

\paragraph{Transformation of problem instance}

The input to the TSP is a graph G and a value B, similar to LazyTSP.
As part of the transformation, m and k is each set to 1, and n is simply
the number of nodes in G. Transforming the graph G to the edge-weight matrix WM
takes polynomial time.

Now, as part of the transformation, the problem instance is extended.
A node r1 is added to WM, at position 2, and r1 have special weights:
The weight between the first node fn and r1 is 1, and the weights between r1 and all nodes
except fn is the same as the weights from fn to all nodes except r1 and fn.
B is furthermore increased by 1. This simulates that the node after fn will
have to be visited, since m is equal to one.
To handle k, another node r2 is added to the end, and the weight between r2 and all other
nodes is infinity. This ensures that r2 will never be added, and handles that
k=1 nodes will not be picked for the tour.

\paragraph{Preservance of YES-NO property}

To prove that LazyTSP is NP-complete, it must be shown that the
transformation of TSP problem instance X preserves the YES-NO property
when transformed into LazyTSP problem instance X2 = T(X).
In order to show this, it will first be shown that
if the problem instance X gives YES, then T(X) also gives YES,
after which it will be shown that if T(X) gives YES, then X also gives YES.

\textbf{If X gives YES, then T(X) gives YES}

Assume that the problem instance X gives YES, and thus have a solution
for TSP. This means that there is a cycle in G that covers all nodes
once and sums up to at most B. This cycle will be noted by the indexes,
\[s = fn, i_2, .., i_n\], where \(i_n\) is the index of the node just before \(i_1\)
in the cycle.

The transformation will add two nodes, and the new cycle solution will include
the first:
\[s' = fn, r1, (i_2+1), (i_3+1), .., (i_n+1) = fn, r1, i'_2, i'_3, .., i'_n\].
Note that 1 has been added to the indexes to accomodate that r1 was inserted into WM.
The number of nodes n' is equal to n+2, and B' is equal to B+1.
The edge weight between fn and r1 is 1, and the edge weight between r1 and
\(i'_2\) is equal to w(fn, \(i_'2\)).

Now, the first part of the left-hand side of \(\star\), the sum of m, becomes equal to 1.
The second sum in \(\star\) is the sum of the nodes from
m+1 to n'-k. This becomes the range 2 to n+1 in s', which fits with
\[w(r1, i'_2) + w(i'_2, i'_3) + .. + w(i'_{n-1}, i'_n)\].
But since \(w(r1, i'_2)\) is equal to w(fn, \(i_'2\)), this
sum is equal to the sum of s taken as a path.
The third part of \(\star\) is simply equal to \(w(i_n, fn)\).
Taken together, the value thus becomes equal to \(w(s') \leq B + 1 = B'\),
fulfilling the constraint \(\star\). The sequence solution to T(X) is then simply the indexes
\(r1, i'_2, i'_3, .., i'_n\).
It has then been shown how to transform a solution in X to a solution in
T(X), and thus preserving that if X gives YES, then T(X) gives YES.

\textbf{If T(X) gives YES, then X gives YES}

Assume that the transformed problem instance T(X) gives YES, and thus
have a solution for LazyTSP. This means that there is a cycle in G'
that fulfils \(\star\). Note this cycle by
\[s' = fn, r1, i'_2, i'_3, .., i'_n\].
Now, the sum of this is equal to \(w(s') \leq B + 1 = B'\). If r1 is removed from the cycle,
a link can instead be made between fn and \(i'_2\), giving: 
\[s = fn, i_2, .., i_n\].
This simply gives a decrease in weight of the cycle of 1, since w(fn, r1) is 1, and
\(w(fn, i'_2) = w(r1, i'_2)\). This results in a weight of the cycle
\(w(s) \leq B'-1 = B\). In the original graph, this means that all nodes are covered,
and that the cycle has weight less than B, which causes X to give YES.
It has then been shown that if T(X) gives YES, then X gives YES.

\paragraph{Conclusion}

Since it has been shown that there exists a polynomial-time algorithm
that transforms problem instances from TSP to LazyTSP, that the
YES-NO property of problem instances is preserved in the transformation,
and that LazyTSP is in NP, it then follows that LazyTSP is NP-complete.

\subsection{Considerations regarding complexity}

First k will be considered, followed by m.

\textbf{k}

Given the constant a, it follows that for those problem instances
for which a does not break the constraints on n, m and k, the difference
between k and n will be constant. k can have an impact on the complexity,
since k nodes does not have to be picked. If the difference between k and n
is constant, then the maximum number of nodes that have to be picked will
also be constant. This gives in effect a problem instance where for increasing n,
a constant number c of nodes will have to be picked from these, fulfilling \(\star\).
Ignoring m, this gives a=c, and a number of different combinations of nodes to pick equal to
\[\frac{n!}{(n-a)!} = n * (n-1) * (n-2) * .. * (n-a+1) \leq n_1 * n_2 * n_3 * .. * n_{a + 2} = n^{a}\]
Note that this formula is similar to the binomial coefficient, but not the same,
since the order of the picked nodes matters.

Since a is constant, this means that the time complexity is polynomial.

\textbf{m}

Given the constant a, it follows that for those problem instances
for which a does not break the constraints on n, m and k, the difference
between m and n will be constant. m has a major impact on the complexity,
since the m+1 first nodes does not have to be considered when finding a solution.
If the difference between m and n then is constant, the number of nodes that will
have to be considered is then also constant. This makes the time complexity polynomial.

