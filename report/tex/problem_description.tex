
\section{a) Problem description}

\textit{Read and understand the problem. Describe in colloquial terms what the problem
is about and explain the main differences to the classical TSP.}

\subsection{Description of the problem}

The input consists of 3 positive integers k, m and B,
and a graph G. The graph G is described by its size n as well
as an edge-weight matrix WM with positive integer weights.
The numbers k and m are constrained by the relation:

\[n - k - m \geq 2\]

A problem instance PI gives YES if there is a cycle C in G
with a specific form, and has a weight w(C) which is at most B,
and NO otherwise.
The form of the cycle can be described in two parts.

The first part of the cycle is the first m nodes in WM.
The second part tours a number of different nodes \(i_1 .. i_{n-k-m}\).
None of \(i_2 .. i_{n-k-m-1\) is contained in the m first nodes.

The number of edges in C, equal to the number of nodes \(|C|\),
is described by \(\star\). It is seen that the number of edges is equal to
\(m\) plus \(n - k - m - 1\) plus 1, which gives:
\[|C| = n - k\]

A possible modelling for the LazyTSP problem is that a lazy salesperson has to visit
a required number of cities \(n - k\), starting and ending in city 1.
She starts off lazily with simply visiting the first m+1 (starting with city 1)
cities on her list of cities, not thinking about how time-effective the route is.
Once in the m+1'th city, she discovers that she has a deadline to meet.
Now, does there exist at least one route that visits the remaining number of cities
and ends in the first city before the deadline ends? Note that the
poor salesperson may already have exceeded the deadline during her lazy route.

\subsection{Comparison with Travelling Salesperson}

If m and k could be set to zero, the problem would simply be TSP,
since the number of cities that must be visited \(n - k\) would become equal
to n, which means that the cycle should visit every single node, and not have any duplicate
nodes.

Given the similarity with TSP, some of the techniques used for TSP may be useful
for LazyTSP. Care should be taken, however, since k is never zero, and thus not all
cities have to be visited. This means that some cities must be picked for the solution,
while others may not be picked, which is different from TSP, where it is known that all
cities must be picked at some point.

